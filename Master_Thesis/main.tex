\documentclass[12pt]{article}
\usepackage[utf8]{inputenc}
\usepackage[a4paper,top=2.1cm,bottom=2.1cm,left=2.4cm,right=2.4cm]{geometry}
\usepackage{graphicx}
\usepackage{setspace}
\usepackage{tabularx}
\usepackage[pdfpagelabels]{hyperref}

\begin{document}

\pagenumbering{arabic}

\begin{titlepage}
\Large
\includegraphics[width=0.4\textwidth]{logo.png}
\renewcommand{\thepage}{Title}
\thispagestyle{empty}
   \begin{center}
       \vspace*{1cm}
\linespread{1.25}
       {\doublespacing \Huge\textbf{Cloud deployment of SAT- \& optimisation-solvers using DevSecOps \& microservices with algorithm selection \& expansion of knowledge base.}}
\linespread{1}
       \rule{\linewidth}{1pt}
       {\huge Master Thesis \\
       \Large Department of Mathematics and Computer Science, \\
       University of Southern Denmark}
\end{center}
\vspace{5cm}
\Large
\begin{tabularx}{\textwidth}{lXr}
Author & & Danni Kiskov Møller \\ \\
\end{tabularx}
\begin{tabularx}{\textwidth}{lXr}
Supervisor & & Jacopo Mauro\\ \\
\end{tabularx}

\vfill
\large \today
\end{titlepage}

\thispagestyle{empty}

\include{0-abstract}

\thispagestyle{empty}
\tableofcontents
\thispagestyle{empty}
\newpage
\stepcounter{page}

\renewcommand{\abstractname}{Abstract}
\begin{abstract}
Todo
\end{abstract}

\begin{center} \bf Keywords \end{center}

\thispagestyle{empty}
\tableofcontents
\thispagestyle{empty}
\newpage
\stepcounter{page}

\section{Introduction}
Traditionally, cloud deploying of solvers for SAT and optimisation tasks involves a tedious process of installation, configuration, and integration with existing software ecosystems.  This process can be time consuming, error-prone, and presents a significant barrier to entry for researchers and students who wish to leverage these tools.  Using DevSecOps, this project aims to design, develop, and evaluate a setup that enables a nearly one-click deployment of SAT and optimisation solvers on cloud infrastructures (Ucloud) using microservice architecture.  This will be designed and developed with regards to best practices in cybersecurity and secure software development on a platform that will process minizinc, CNF and SMTlib files, supporting containerised deployment of solvers.\\
\\
Next,  this  project  will  allow  for  smart  algorithm  selection  using  algorithm  selection  techniques such as Sub-portfolio Nearest Neighbour Lazy algorithm portfolio (SUNNY) \& Algorithm Selection Library (ASlib):  With a multitude of solver algorithms available, each with distinct strengths and weaknesses,  the process of manually selecting the most suitable algorithm can be rise challenges. The goal of this part is to integrate algorithm selection capabilities.\\
\\
Finally, the project will foster the training of the artificial intelligence behind the selection of solvers by expanding the knowledge base of solved instances for the algorithm selectors to choose from.
\section{Keywords}
security: docker-in-docker vs singularity\\
terraform to deploy locally
\bibliography{biblio.bib}
\bibliographystyle{plain}

% appendix if needed

\end{document}
